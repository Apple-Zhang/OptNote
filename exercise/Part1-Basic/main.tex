\documentclass[UTF8]{ctexart}
\usepackage{amsmath}
\usepackage{amsthm}
\usepackage{amsfonts}
\usepackage{amssymb}


\usepackage{bm}

\theoremstyle{definition}
\newtheorem{problem}{题目}
\newenvironment{solution}{\begin{proof}[\indent\bf 解]}{\end{proof}}

\newcommand{\onef}{\bm 1}

% upper bold face
\newcommand{\af}{\mathbf a}
\newcommand{\bb}{\mathbf b}
\newcommand{\A}{\mathbf A}
\newcommand{\D}{\mathbf D}
\newcommand{\Z}{\mathbf Z}
\newcommand{\X}{\mathbf X}
\newcommand{\Ibf}{\mathbf I}
\newcommand{\Q}{\mathbf Q}
\newcommand{\W}{\mathbf W}
\newcommand{\M}{\mathbf M}

% lower bold face
\newcommand{\uf}{\mathbf u}
\newcommand{\x}{\mathbf x}
\newcommand{\z}{\mathbf z}
\newcommand{\w}{\mathbf w}
\newcommand{\y}{\mathbf y}
\newcommand{\q}{\mathbf q}
\newcommand{\vf}{\mathbf v}

% bold greek
\newcommand{\bxi}{\bm\xi}
\newcommand{\balpha}{\bm\alpha}
\newcommand{\bmu}{\bm\mu}
\newcommand{\bLambda}{\bm\Lambda}
\newcommand{\bphi}{\bm\phi}

% mathcal
\newcommand{\Dc}{\mathcal D}
\newcommand{\Yc}{\mathcal Y}
\newcommand{\Ic}{\mathcal I}
\newcommand{\Vc}{\mathcal V}

% mathbb
\newcommand{\Rb}{\mathbb R}

% special notations
\newcommand{\st}{\text{s.t. }}
\newcommand{\derv}[2]{\nabla_{#2}{#1}}
\newcommand{\partf}[2]{\frac{\partial #1}{\partial #2}}
\newcommand{\partff}[3]{\frac{\partial^2#1}{\partial #2\partial #3}}
\newcommand{\sgn}{\mathrm{sign}}
\newcommand{\rk}{\mathrm{rank}}
\newcommand{\diag}{\mathrm{diag}}
\newcommand{\tr}{tr}
\newcommand{\dom}{\mathrm {dom~}}

\title{第一部分:基本理论}
\author{Apple Zhang}
\date{\today}

\begin{document}
\maketitle
\newpage

\begin{problem}
  证明 $f(\x)=(x_1x_2\cdots x_n)^{1/n}, \x\in\Rb^{n}_{++}$ 是凹函数.
\end{problem}
\begin{proof}
  显然定义域是一个凸锥,且为凸集.
  因此其梯度
  \begin{equation}
    \partf{f}{x_i}=\frac{f(\x)}{nx_i}
  \end{equation}
  以及海森矩阵
  \begin{equation}
    \frac{\partial^2 f}{\partial x_i\partial x_j}=\frac{f(\x)}{n^2x_ix_j},\quad
    \frac{\partial^2 f}{\partial x_i^2}=-(n-1)\frac{f(\x)}{n^2x_i^2},\quad i\not=j.
  \end{equation}
  或写成
  \begin{equation}
    \nabla^2 f(\x)=\frac{f(\x)}{n^2}\left[\z\z^T-n\diag\left(\frac1{x_1^2}, \frac1{x_2^2}, \cdots, \frac{1}{x^2_n}\right)\right].
  \end{equation}
  其中,$\z=[1/x_1,1/x_2,\cdots,1/x_n]^T$.
  则对于任意的非零向量 $\w\in\Rb^n$,有
  \begin{align*}
    \w^T\nabla^2f(\x)\w
    &=\frac{f(\x)}{n^2}\left[\w^T\z\z^T\w-n\w^T\diag\left(\frac1{x_1^2}, \frac1{x_2^2}, \cdots, \frac{1}{x^2_n}\right)\w\right]\\
    &=\frac{f(\x)}{n^2}\left[\left(\sum_{i=1}^nw_iz_i\right)^2-n\sum_{i=1}^nw_i^2z_i^2\right]\\
    &\leq 0.
  \end{align*}
  其中最后一步用到了柯西不等式,即任意向量 $\af, \bb$,都有
  \begin{equation}
    (\af^T\bb)^2\leq(\af^T\af)(\bb^T\bb).
  \end{equation}
  此处取向量 $a_i=1$ 和 $b_i=u_iz_i$ 即可.
\end{proof}
\begin{problem}
  设 $f:\Rb^n\mapsto\Rb$,$g:\Rb^n\times \Rb_{++}\mapsto\Rb$,其中
  \begin{equation}
    g(\x,t)=tf\left(\frac{\x}{t}\right).
  \end{equation}
  且 $\dom g=\{(\x,t):t>0,~\x/t\in\dom f\}$. 则称 $g$ 是关于函数 $f$ 的透视.
  证明如果 $f$ 是凸函数,则 $g$ 也是凸的. 同时,若 $f$ 为凹函数,则 $g$ 也是凹函数.
\end{problem}
\begin{proof}
  如果 $f$ 是凸函数,则显然 $\dom f$ 是凸集,注意到 $\dom g=\{(\x,t):t>0,x/t\}$ 是 $\dom f$ 在透视函数下的原相,所以 $\dom g$ 也是凸集.
  
  对于任意 $t_1,t_2>0$, $\x_1,\x_2\in\dom f$,$\theta\in[0,1]$,并且 $\x_1/t_1, \x_2/t_2\in \dom f$,有
  \begin{align*}
    &~g(\theta(\x_1,t_1)+(1-\theta)(\x_2,t_2))\\
   =&~g(\theta\x_1+(1-\theta)\x_2, \theta t_1+(1-\theta)t_2)\\
   =&~[\theta t_1+(1-\theta)t_2]f\left(\frac{\theta\x_1+(1-\theta)\x_2}{\theta t_1+(1-\theta)t_2}\right)\\
   =&~[\theta t_1+(1-\theta)t_2]f\left(\frac{\theta t_1\x_1/t_1+(1-\theta)t_2\x_2/t_2}{\theta t_1+(1-\theta)t_2}\right)\\
\leq&~[\theta t_1+(1-\theta)t_2]\left[\frac{\theta t_1}{\theta t_1+(1-\theta)t_2}f\left(\frac{\x_1}{t_1}\right)
   +\frac{(1-\theta)t_2}{\theta t_1+(1-\theta)t_2}f\left(\frac{\x_2}{t_2}\right)\right]\\
   =&~\theta t_1f\left(\frac{\x_1}{t_1}\right)+(1-\theta) t_2f\left(\frac{\x_2}{t_2}\right)\\
   =&~\theta g(\x_1,t_1)+(1-\theta)g(\x_2,t_2).
  \end{align*}
  其中第 5 行用到了 $f$ 为凸函数.
  根据以上及凸函数的定义,可知 $g$ 是凸的. 凹函数的情况完全可以类似证明.
\end{proof}
\begin{problem}[Boyd 2.1]
  Let $C\subseteq\Rb^n$ be a convex set, with $\x_1,\cdots,\x_k\in C$,
  and let $\theta_1,\cdots,\theta_k\in\Rb$ satisfy $\theta_i\geq 0$,
  $\theta_1+\cdots+\theta_k=1$. Show that $\theta_1\x_1+\cdots+\theta_k\x_k\in C$.
  (The definition of convexity is that this holds for $k=2$; you must show it for arbitrary $k$.) \emph{Hint}: Use induction on $k$.
\end{problem}
\begin{proof}
  当 $k=2$ 时,根据凸集的性质,结论显然成立.
  假设当 $k=l$ 时结论成立,即对于任意 $\x_1,\cdots,\x_l\in C$ 和 $\theta_i\geq 0, \theta_1+\cdots+\theta_l=1$,有
  \begin{equation}
    \theta_1\x_1+\theta_2\x_2+\cdots+\theta_l\x_l\in C.
  \end{equation}
  则当 $k=l+1$ 时,任意取 $\x_1,\cdots,\x_{l+1}\in C$ 和 $\theta_i\geq 0, \theta_1+\cdots+\theta_{l+1}=1$,
  由于我们知道
  \begin{equation}
    % \frac{\theta_1\x_1}{\theta_1+\theta_2+\cdots+\theta_l}+\cdots+\frac{\theta_l\x_l}{\theta_1+\theta_2+\cdots+\theta_l}=
    \sum_{j=1}^{l}\frac{\theta_j\x_j}{\sum_{i=1}^l\theta_i}\in C.
  \end{equation}
  因此,我们有
  \begin{align*}
     &~\theta_1\x_1+\theta_2\x_2+\cdots+\theta_{l+1}\x_{l+1}\\
    =&~\theta_1\x_1+\theta_2\x_2+\cdots+\left(1-\sum_{i=1}^l \theta_i\right)\x_{l+1}\\
    =&~\left(\sum_{i=1}^l\theta_i\right)\left(\sum_{j=1}^{l}\frac{\theta_j\x_j}{\sum_{i=1}^l\theta_i}\right)+\left(1-\sum_{i=1}^l \theta_i\right)\x_{l+1}\\
  \in&~C.
  \end{align*}
  根据数学归纳法,结论得证.
\end{proof}

% TODO (some alternative)
% 2.1, 2.2, 2.5, 2.7, 2.10, 2.16, 2.18, 2.19
% 3.1, 3.2, 3.5, 3.13, 3.18, 3.21, 3.32, 3.33, 3.36, 3.43
\end{document}